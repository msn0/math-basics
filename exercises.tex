\documentclass[a4paper,12pt,leqno,fleqn]{article}
\usepackage[polish]{babel}
\usepackage[utf8]{inputenc}
\usepackage[T1]{fontenc}
\usepackage{amsmath}
\usepackage{tabularx}
\usepackage{multicol}

\def\tabularxcolumn#1{m{#1}}
\begin{document}

Usuń niewymierność z mianownika

\begin{tabularx}{\linewidth}{@{}XX@{}XX@{}XX@{}}

  \begin{equation}
    \frac{1}{\sqrt{2}}
  \end{equation}
&
  \begin{equation}
    \frac{1}{\sqrt{3}}
  \end{equation}
&
  \begin{equation}
    \frac{1}{\sqrt{7}}
  \end{equation}
&
  \begin{equation}
    \frac{1}{\sqrt{4}}
  \end{equation}
\\
  \begin{equation}
    \frac{\sqrt{2}}{\sqrt{3}}
  \end{equation}
&
  \begin{equation}
    \frac{\sqrt{2}+1}{\sqrt{5}}
  \end{equation}
&
  \begin{equation}
    \frac{\sqrt{2}+\sqrt{3}}{\sqrt{7}}
  \end{equation}
&
  \begin{equation}
    \frac{\sqrt{7}-2}{\sqrt{2}}
  \end{equation}
\\
  \begin{equation}
    \frac{2-\sqrt{11}}{\sqrt{3}}
  \end{equation}
&
  \begin{equation}
    \frac{3\sqrt{3}}{\sqrt{2}}
  \end{equation}
&
  \begin{equation}
    \frac{3\sqrt{7}-2\sqrt{2}}{3\sqrt{2}}
  \end{equation}
&
  \begin{equation}
    \frac{-\sqrt{7}-1}{-\sqrt{2}}
  \end{equation}
\\
  \begin{equation}
    \frac{1}{\sqrt{2}+1}
  \end{equation}
&
  \begin{equation}
    \frac{1}{\sqrt{2}-1}
  \end{equation}
&
  \begin{equation}
    \frac{\sqrt{2}}{\sqrt{5}-1}
  \end{equation}
&
  \begin{equation}
    \frac{3\sqrt{2}}{\sqrt{7}+3}
  \end{equation}
\\
  \begin{equation}
    \frac{2\sqrt{3}-4}{1-\sqrt{2}}
  \end{equation}
&
  \begin{equation}
    \frac{\tfrac{1}{2}+\sqrt{2}}{\sqrt{2}-2}
  \end{equation}
&
  \begin{equation}
    \frac{\sqrt{2}-\sqrt{3}}{\sqrt{5}-\sqrt{7}}
  \end{equation}
&
  \begin{equation}
    \frac{\sqrt{3}-3\sqrt{2}}{\sqrt{7}+2\sqrt{3}}
  \end{equation}

\end{tabularx}

\newpage
Oblicz

\begin{tabularx}{\linewidth}{@{}XX@{}}

  \begin{equation}
    \left(\frac{3+\sqrt{2}}{2}\right)^2
  \end{equation}
&
  \begin{equation}
    \left(\frac{1+\sqrt{3}}{\sqrt{2}}\right)^2
  \end{equation}
\\
  \begin{equation}
    3\cdot\sqrt[3]{27}+3
  \end{equation}
&
  \begin{equation}
    \frac{1}{8}\left(4-\frac{\sqrt{2}}{3}\right)^2
  \end{equation}
\\
  \begin{equation}
    -\left[\sqrt{2}-\sqrt{3}\cdot(\sqrt{6}-1)\right]
  \end{equation}
&
  \begin{equation}
    (\sqrt{2}+\sqrt{3})(\sqrt{2}-\sqrt{3})
  \end{equation}
\\
  \begin{equation}
    \left(3\cdot3^{\frac{1}{2}}+3^{\frac{1}{3}}\right)\cdot\sqrt{3}
  \end{equation}
&
  \begin{equation}
    \sqrt[7]{3}\cdot3^{\frac{1}{7}}\cdot\left(\sqrt{3}\cdot\sqrt{2}-\sqrt{6}\right)
  \end{equation}
\\
  \begin{equation}
    3^7\cdot3^{-7}\cdot3^3
  \end{equation}
&
  \begin{equation}
    2^5\cdot2^{-4}-4^7:4^6
  \end{equation}
\\
  \begin{equation}
    2^7\cdot4^{-3}+3^{14}:\left(\frac{1}{3}\right)^{-15}
  \end{equation}
&
  \begin{equation}
    1-\frac{1}{3^{2}}+3\cdot\sqrt{3}\cdot3^{\frac{1}{2}}
  \end{equation}
\\
  \begin{equation}
    \frac{-2+16^{-\frac{1}{2}}}{\sqrt{2}}
  \end{equation}
&
  \begin{equation}
    \frac{3^{13}}{9^{5}}\cdot9
  \end{equation}
\\
  \begin{equation}
    \frac{2\cdot2^{7}:\left(\frac{1}{4}\right)^{-2}}{2^{7}}
  \end{equation}
&
  \begin{equation}
    \frac{\sqrt[3]{9}\cdot3^{-2}\cdot{\sqrt[4]{3}}}{3}
  \end{equation}

\end{tabularx}

\newpage
Narysuj wykres funkcji

\begin{tabularx}{\linewidth}{@{}XX@{}XX@{}}

  \begin{equation}
    y=x
  \end{equation}
&
  \begin{equation}
    y=2x
  \end{equation}
&
  \begin{equation}
    y=3x
  \end{equation}
\\
  \begin{equation}
    y=\frac{1}{2}x
  \end{equation}
&
  \begin{equation}
    y=\frac{1}{3}x
  \end{equation}
&
  \begin{equation}
    y=x+1
  \end{equation}
\\
  \begin{equation}
    y=x+2
  \end{equation}
&
  \begin{equation}
    y=x-2
  \end{equation}
&
  \begin{equation}
    y=2x-1
  \end{equation}
\\
  \begin{equation}
    y=3x-2
  \end{equation}
&
  \begin{equation}
    f(x)=2-x
  \end{equation}
&
  \begin{equation}
    f(x)=\frac{1}{2}x-2
  \end{equation}
\\
  \begin{equation}
    y=2x-4
  \end{equation}
&
  \begin{equation}
    y=-x
  \end{equation}
&
  \begin{equation}
    y=-2x
  \end{equation}
\\
  \begin{equation}
    y=-3x-2
  \end{equation}
&
  \begin{equation}
    y=-\frac{1}{2}x-4
  \end{equation}
&
  \begin{equation}
    y=2
  \end{equation}
\\
  \begin{equation}
    g(t)=3t+1
  \end{equation}
&
  \begin{equation}
    x\mapsto x-2
  \end{equation}
&
  \begin{equation}
    t\mapsto -2-2t
  \end{equation}
\\
  \begin{equation}
    f(x)=0
  \end{equation}
&
  \begin{equation}
    2y=4x-2
  \end{equation}
&
  \begin{equation}
    y=x\cdot\sqrt{2}-1
  \end{equation}
\\
\end{tabularx}

\newpage
Rozwiąż równanie

\begin{tabularx}{\linewidth}{@{}XX@{}}

  \begin{equation}
    -3x=4-2\cdot(3x+2)
  \end{equation}
&
  \begin{equation}
    x^2-2x=4+(x+1)^2
  \end{equation}
\\
  \begin{equation}
    x-(1-2x)=3
  \end{equation}
&
  \begin{equation}
    x-(2x+1)^2=-x-4x^2+3
  \end{equation}
\\
  \begin{equation}
    4x+1=3x-\sqrt{2}
  \end{equation}
&
  \begin{equation}
    1-3x=3\sqrt{2}\cdot x-3
  \end{equation}
\\
  \begin{equation}
    2-\frac{1}{2}(x+7)=\frac{3-x}{2}
  \end{equation}
&
  \begin{equation}
    \frac{2-x}{3}+\frac{x-4}{2}=1
  \end{equation}
\\
  \begin{equation}
    (x+4)(x+3)=x^2-7
  \end{equation}
&
  \begin{equation}
    \frac{4x-2}{2}-\frac{3x-3}{3}=4-x
  \end{equation}
\\
  \begin{equation}
    \frac{2x+7}{3}=\frac{4x-1}{5}
  \end{equation}
&
  \begin{equation}
    \frac{7-3t}{2}-t+4=3t-1
  \end{equation}
\\
  \begin{equation}
    x(x-3)-x^2=0
  \end{equation}
&
  \begin{equation}
    (x+4)(3-x)=-x^2-(2x+1)
  \end{equation}
\\
  \begin{equation}
    \frac{\sqrt{3}}{3}\cdot x(x-3)=\frac{x(6x-1)}{6\sqrt{3}}
  \end{equation}
&
  \begin{equation}
    \frac{x}{\sqrt[3]{2}}\cdot 2^{\frac{4}{3}}=\frac{4^{3}}{2^{2\frac{1}{2}}}-2x
  \end{equation}
\\
\end{tabularx}

\newpage
Zaznacz na osi liczbowej

\begin{tabularx}{\linewidth}{@{}XX@{}XX@{}}

  \begin{equation}
    x\in(1,3)
  \end{equation}
&
  \begin{equation}
    x\in(1,3\rangle
  \end{equation}
&
  \begin{equation}
    x\in\langle1,3\rangle
  \end{equation}
\\
  \begin{equation}
    x\in(-\infty,1\rangle
  \end{equation}
&
  \begin{equation}
    x\in(-\infty,\tfrac{3}{2})
  \end{equation}
&
  \begin{equation}
    x\in(1,3\rangle\cup\langle5,10)
  \end{equation}
\\
\end{tabularx}

\begin{tabularx}{\linewidth}{@{}XX@{}}

  \begin{equation}
    x\in(-1,1)\cup(2,\infty)
  \end{equation}
&
  \begin{equation}
    x\in(-\infty,-1\rangle\cup\langle1,\infty)
  \end{equation}
\\
  \begin{equation}
    x\in(-\infty,1)\cup(1,\infty)
  \end{equation}
&
  \begin{equation}
    x\in\langle1,5\rangle\cap\langle2,10\rangle
  \end{equation}
\\
  \begin{equation}
    x\in(-7,1)\cap(-1,3)
  \end{equation}
&
  \begin{equation}
    x\in(-\infty,5\rangle\cup(2,7)
  \end{equation}
\\
  \begin{equation}
    x\in(-3,1\rangle\cup(1,\infty)
  \end{equation}
&
  \begin{equation}
    x\in(-\infty,3)\cap\langle1,\infty)
  \end{equation}
\\
  \begin{equation}
    x\in(-7,1)\cap(5,\infty)
  \end{equation}
&
  \begin{equation}
    x\in(0,4)\cup\{6\}
  \end{equation}
\\
  \begin{equation}
    x\in(-1,1)\cap\{0\}
  \end{equation}
&
  \begin{equation}
    x\in(-7,7)\setminus\langle0,10\rangle
  \end{equation}
\\
\end{tabularx}

\newpage
Rozwiąż nierówność

\begin{tabularx}{\linewidth}{@{}XX@{}}

  \begin{equation}
    2x-6\leq x
  \end{equation}
&
  \begin{equation}
    x-\tfrac{1}{2}\leq\tfrac{3}{4}-x
  \end{equation}
\\
  \begin{equation}
    -3x-(3-x)\geq-x+3
  \end{equation}
&
  \begin{equation}
    3x-\frac{x-2}{2}\leq x
  \end{equation}
\\
  \begin{equation}
    \frac{x+3}{3}+\frac{x+2}{2}<0
  \end{equation}
&
  \begin{equation}
    \frac{2x-3}{4}>\frac{x+2}{2}
  \end{equation}
\\
  \begin{equation}
    \frac{2x-3}{4}\geq-\frac{-x-2}{2}
  \end{equation}
&
  \begin{equation}
    -\frac{3-x}{2}\geq\frac{2x-1}{4}
  \end{equation}
\\
  \begin{equation}
    x+\frac{-2x+3}{4}\leq-2\cdot\frac{2+x}{2}
  \end{equation}
&
  \begin{equation}
    |x-1|<3
  \end{equation}
\\
  \begin{equation}
    |x+7|\leq3
  \end{equation}
&
  \begin{equation}
    3-|x+1|<0
  \end{equation}
\\
  \begin{equation}
    |2x-1|\geq3
  \end{equation}
&
  \begin{equation}
    |x|<0
  \end{equation}
\\
  \begin{equation}
    3\cdot|1-x|>1
  \end{equation}
&
  \begin{equation}
    3<|x-3|
  \end{equation}
\\
\end{tabularx}

\newpage
Rozwiąż równanie

\begin{tabularx}{\linewidth}{@{}XX@{}}

  \begin{equation}
    x^2-3x+2=0
  \end{equation}
&
  \begin{equation}
    x^2-2x+1=0
  \end{equation}
\\
  \begin{equation}
    2x^2-5x+2=0
  \end{equation}
&
  \begin{equation}
    x^2-8x+15=0
  \end{equation}
\\
  \begin{equation}
    x^2+2x+1=0
  \end{equation}
&
  \begin{equation}
    x^2+3x+2=0
  \end{equation}
\\
  \begin{equation}
    x^2-6x+9=0
  \end{equation}
&
  \begin{equation}
    2x^2-2x=0
  \end{equation}
\\
  \begin{equation}
    9x+3x^2-12=0
  \end{equation}
&
  \begin{equation}
    2x(x-4)=0
  \end{equation}
\\
  \begin{equation}
    (x-3)(x+4)=0
  \end{equation}
&
  \begin{equation}
    2x^2+18=12x
  \end{equation}
\\
  \begin{equation}
    -5x+2x^2=-3
  \end{equation}
&
  \begin{equation}
    \frac{-x^2+x}{2}+x=3
  \end{equation}
\\
  \begin{equation}
    \frac{x-3}{x}-x+4=0
  \end{equation}
&
  \begin{equation}
    \frac{x-1}{x+1}=\frac{x-2}{x+2}
  \end{equation}

\end{tabularx}

\newpage
Rozwiąż nierówność

\begin{tabularx}{\linewidth}{@{}XX@{}}


  \begin{equation}
    x^2-1>0
  \end{equation}
&
  \begin{equation}
    2+x^2-3x>0
  \end{equation}
\\
  \begin{equation}
    x^2+6\leq5x
  \end{equation}
&
  \begin{equation}
    x^2+x-2<0
  \end{equation}
\\
  \begin{equation}
    x^2+2x+3\geq0
  \end{equation}
&
  \begin{equation}
    x(x-3)<0
  \end{equation}
\\
  \begin{equation}
    -2x^2+2x-7>0
  \end{equation}
&
  \begin{equation}
    2x^2-4x-6<0
  \end{equation}
\\
  \begin{equation}
    x^2-2x+2>0
  \end{equation}
&
  \begin{equation}
    x^2+4x+4\leq0
  \end{equation}
\\
  \begin{equation}
    3x^2-8x+5<0
  \end{equation}
&
  \begin{equation}
    2(x-3)(x+2)\leq0
  \end{equation}
\\
  \begin{equation}
    3x^2-x+2\geq0
  \end{equation}
&
  \begin{equation}
    -x^2-x-2\leq0
  \end{equation}
\\
  \begin{equation}
    \frac{(x-1)^2}{3}-(x-1)\geq0
  \end{equation}
&
  \begin{equation}
    \frac{(x-1)^2}{2}+\left(\frac{x-4}{2}\right)^2\leq0
  \end{equation}

\end{tabularx}

\newpage
Rozwiąż równanie

\begin{tabularx}{\linewidth}{@{}XX@{}}

  \begin{equation}
    x^3-x^2-x+1=0
  \end{equation}
&
  \begin{equation}
    x^3+2x^2-x-2=0
  \end{equation}
\\
  \begin{equation}
    x^3-2x^2-x+2=0
  \end{equation}
&
  \begin{equation}
    x^3+x^2-9x-9=0
  \end{equation}
\\
  \begin{equation}
    -4x^3+4x^2+4x-4=0
  \end{equation}
&
  \begin{equation}
    16x^3-4x^2-4x+1=0
  \end{equation}
\\
  \begin{equation}
    3x^3-6x^2-48x+96=0
  \end{equation}
&
  \begin{equation}
    x^3+3x^2-9x-27=0
  \end{equation}
\\
  \begin{equation}
    x^3-2x^2-3x+6=0
  \end{equation}
&
  \begin{equation}
    2x^3-8x^2-8x+32=0
  \end{equation}
\\
  \begin{equation}
    x^3-x^2-x-1=0
  \end{equation}
&
  \begin{equation}
    x^3+x^2+x+1=0
  \end{equation}
\\
  \begin{equation}
    x^3-2x^2+4x-8=0
  \end{equation}
&
  \begin{equation}
    4x^3+2x^2-2x-1=0
  \end{equation}
\\
  \begin{equation}
    x^3-7x^2-7x+49=0
  \end{equation}
&
  \begin{equation}
    3x^3-x=0
  \end{equation}

\end{tabularx}

\newpage
Znajdź różnicę $r$ oraz $n$-ty wyraz ciągu arytmetycznego $a_n$ mając dane

\begin{tabularx}{\linewidth}{@{}XX@{}}

  \begin{equation}
    a_2=3,\:a_6=11,\:n=4
  \end{equation}
&
  \begin{equation}
    a_3=1,\:a_7=9,\:n=1
  \end{equation}
\\
  \begin{equation}
    a_4=3,\:a_9=9,\:n=5
  \end{equation}
&
  \begin{equation}
    a_1=30,\:a_8=90,\:n=3
  \end{equation}
\\
  \begin{equation}
    a_7=21,\:a_{14}=3,\:n=5
  \end{equation}
&
  \begin{equation}
    a_{14}=60,\:a_{19}=35,\:n=10
  \end{equation}
\\
  \begin{equation}
    a_2=3,\:a_7=-12,\:n=5
  \end{equation}
&
  \begin{equation}
    a_{13}=100,\:a_{17}=90,\:n=10
  \end{equation}

\end{tabularx}

\vspace{20pt}
Znajdź $n$-ty wyraz ciągu arytmetycznego $a_n$ mając dane

\begin{tabularx}{\linewidth}{@{}XX@{}}

  \begin{equation}
    a_2=3,\:r=11,\:n=4
  \end{equation}
&
  \begin{equation}
    a_6=1,\:r=9,\:n=1
  \end{equation}
\\
  \begin{equation}
    a_4=3,\:r=4,\:n=10
  \end{equation}
&
  \begin{equation}
    a_1=32,r=\frac{1}{2},\:n=7
  \end{equation}
\\
  \begin{equation}
    a_7=-10,\:r=3,\:n=4
  \end{equation}
&
  \begin{equation}
    a_{11}=60,r=\frac{3}{4},\:n=16
  \end{equation}
\\
  \begin{equation}
    a_3=-\frac{2}{3},\:r=3,\:n=9
  \end{equation}
&
  \begin{equation}
    a_{12}=-60,r=-15,\:n=18
  \end{equation}

\end{tabularx}

\newpage
Znajdź iloraz $q$ oraz $n$-ty wyraz ciągu geometrycznego $a_n$ mając dane

\begin{tabularx}{\linewidth}{@{}XX@{}}

  \begin{equation}
    a_2=3,\:a_6=48,\:n=8
  \end{equation}
&
  \begin{equation}
    a_3=2,\:a_{18}=9,\:n=2
  \end{equation}
\\
  \begin{equation}
    a_4=8,\:a_7=27,\:n=10
  \end{equation}
&
  \begin{equation}
    a_1=60,\:a_4=7\frac{1}{2},\:n=5
  \end{equation}
\\
  \begin{equation}
    a_7=21,\:a_{10}=\frac{7}{9},\:n=5
  \end{equation}
&
  \begin{equation}
    a_{10}=600,\:a_{13}=22\frac{2}{9},\:n=8
  \end{equation}
\\
  \begin{equation}
    a_2=4,\:a_6=128,\:n=9
  \end{equation}
&
  \begin{equation}
    a_2=1,\:a_{5}=16,\:n=7
  \end{equation}
\\
\end{tabularx}


\end{document}
