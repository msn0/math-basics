\documentclass[a4paper,12pt,leqno,fleqn]{article}
\usepackage[polish]{babel}
\usepackage[utf8x]{inputenc}
\usepackage{amsmath}
\usepackage{tabularx}
\usepackage{multicol}

\begin{document}

Usuń niewymierność z mianownika

\begin{tabularx}{\linewidth}{@{}XX@{}XX@{}XX@{}}

  \begin{equation}
    \frac{1}{\sqrt{2}}
  \end{equation}
&
  \begin{equation}
    \frac{1}{\sqrt{3}}
  \end{equation}
&
  \begin{equation}
    \frac{1}{\sqrt{7}}
  \end{equation}
&
  \begin{equation}
    \frac{1}{\sqrt{4}}
  \end{equation}
\\
  \begin{equation}
    \frac{\sqrt{2}}{\sqrt{3}}
  \end{equation}
&
  \begin{equation}
    \frac{\sqrt{2}+1}{\sqrt{5}}
  \end{equation}
&
  \begin{equation}
    \frac{\sqrt{2}+\sqrt{3}}{\sqrt{7}}
  \end{equation}
&
  \begin{equation}
    \frac{\sqrt{7}-2}{\sqrt{2}}
  \end{equation}
\\
  \begin{equation}
    \frac{2-\sqrt{11}}{\sqrt{3}}
  \end{equation}
&
  \begin{equation}
    \frac{3\sqrt{3}}{\sqrt{2}}
  \end{equation}
&
  \begin{equation}
    \frac{3\sqrt{7}-2\sqrt{2}}{3\sqrt{2}}
  \end{equation}
&
  \begin{equation}
    \frac{-\sqrt{7}-1}{-\sqrt{2}}
  \end{equation}
\\

  \begin{equation}
    \frac{1}{\sqrt{2}+1}
  \end{equation}
&
  \begin{equation}
    \frac{1}{\sqrt{2}-1}
  \end{equation}
&
  \begin{equation}
    \frac{\sqrt{2}}{\sqrt{5}-1}
  \end{equation}
&
  \begin{equation}
    \frac{3\sqrt{2}}{\sqrt{7}+3}
  \end{equation}
\\
  \begin{equation}
    \frac{2\sqrt{3}-4}{1-\sqrt{2}}
  \end{equation}
&
  \begin{equation}
    \frac{\tfrac{1}{2}+\sqrt{2}}{\sqrt{2}-2}
  \end{equation}
&
  \begin{equation}
    \frac{\sqrt{2}-\sqrt{3}}{\sqrt{5}-\sqrt{7}}
  \end{equation}
&
  \begin{equation}
    \frac{\sqrt{3}-3\sqrt{2}}{\sqrt{7}+2\sqrt{3}}
  \end{equation}


\end{tabularx}

\end{document}