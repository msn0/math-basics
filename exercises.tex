\documentclass[a4paper,12pt,leqno,fleqn]{article}
\usepackage[polish]{babel}
\usepackage[utf8]{inputenc}
\usepackage[T1]{fontenc}
\usepackage{amsmath}
\usepackage{tabularx}
\usepackage{multicol}

\def\tabularxcolumn#1{m{#1}}
\begin{document}

Usuń niewymierność z mianownika

\begin{tabularx}{\linewidth}{@{}XX@{}XX@{}XX@{}}

  \begin{equation}
    \frac{1}{\sqrt{2}}
  \end{equation}
&
  \begin{equation}
    \frac{1}{\sqrt{3}}
  \end{equation}
&
  \begin{equation}
    \frac{1}{\sqrt{7}}
  \end{equation}
&
  \begin{equation}
    \frac{1}{\sqrt{4}}
  \end{equation}
\\
  \begin{equation}
    \frac{\sqrt{2}}{\sqrt{3}}
  \end{equation}
&
  \begin{equation}
    \frac{\sqrt{2}+1}{\sqrt{5}}
  \end{equation}
&
  \begin{equation}
    \frac{\sqrt{2}+\sqrt{3}}{\sqrt{7}}
  \end{equation}
&
  \begin{equation}
    \frac{\sqrt{7}-2}{\sqrt{2}}
  \end{equation}
\\
  \begin{equation}
    \frac{2-\sqrt{11}}{\sqrt{3}}
  \end{equation}
&
  \begin{equation}
    \frac{3\sqrt{3}}{\sqrt{2}}
  \end{equation}
&
  \begin{equation}
    \frac{3\sqrt{7}-2\sqrt{2}}{3\sqrt{2}}
  \end{equation}
&
  \begin{equation}
    \frac{-\sqrt{7}-1}{-\sqrt{2}}
  \end{equation}
\\
  \begin{equation}
    \frac{1}{\sqrt{2}+1}
  \end{equation}
&
  \begin{equation}
    \frac{1}{\sqrt{2}-1}
  \end{equation}
&
  \begin{equation}
    \frac{\sqrt{2}}{\sqrt{5}-1}
  \end{equation}
&
  \begin{equation}
    \frac{3\sqrt{2}}{\sqrt{7}+3}
  \end{equation}
\\
  \begin{equation}
    \frac{2\sqrt{3}-4}{1-\sqrt{2}}
  \end{equation}
&
  \begin{equation}
    \frac{\tfrac{1}{2}+\sqrt{2}}{\sqrt{2}-2}
  \end{equation}
&
  \begin{equation}
    \frac{\sqrt{2}-\sqrt{3}}{\sqrt{5}-\sqrt{7}}
  \end{equation}
&
  \begin{equation}
    \frac{\sqrt{3}-3\sqrt{2}}{\sqrt{7}+2\sqrt{3}}
  \end{equation}

\end{tabularx}

\newpage
Oblicz

\begin{tabularx}{\linewidth}{@{}XX@{}}

  \begin{equation}
    \left(\frac{3+\sqrt{2}}{2}\right)^2
  \end{equation}
&
  \begin{equation}
    \left(\frac{1+\sqrt{3}}{\sqrt{2}}\right)^2
  \end{equation}
\\
  \begin{equation}
    3\cdot\sqrt[3]{27}+3
  \end{equation}
&
  \begin{equation}
    \frac{1}{8}\left(4-\frac{\sqrt{2}}{3}\right)^2
  \end{equation}
\\
  \begin{equation}
    -\left[\sqrt{2}-\sqrt{3}\cdot(\sqrt{6}-1)\right]
  \end{equation}
&
  \begin{equation}
    (\sqrt{2}+\sqrt{3})(\sqrt{2}-\sqrt{3})
  \end{equation}
\\
  \begin{equation}
    \left(3\cdot3^{\frac{1}{2}}+3^{\frac{1}{3}}\right)\cdot\sqrt{3}
  \end{equation}
&
  \begin{equation}
    \sqrt[7]{3}\cdot3^{\frac{1}{7}}\cdot\left(\sqrt{3}\cdot\sqrt{2}-\sqrt{6}\right)
  \end{equation}
\\
  \begin{equation}
    3^7\cdot3^{-7}\cdot3^3
  \end{equation}
&
  \begin{equation}
    2^5\cdot2^{-4}-4^7:4^6
  \end{equation}
\\
  \begin{equation}
    2^7\cdot4^{-3}+3^{14}:\left(\frac{1}{3}\right)^{-15}
  \end{equation}
&
  \begin{equation}
    1-\frac{1}{3^{2}}+3\cdot\sqrt{3}\cdot3^{\frac{1}{2}}
  \end{equation}
\\
  \begin{equation}
    \frac{-2+16^{-\frac{1}{2}}}{\sqrt{2}}
  \end{equation}
&
  \begin{equation}
    \frac{3^{13}}{9^{5}}\cdot9
  \end{equation}
\\
  \begin{equation}
    \frac{2\cdot2^{7}:\left(\frac{1}{4}\right)^{-2}}{2^{7}}
  \end{equation}
&
  \begin{equation}
    \frac{\sqrt[3]{9}\cdot3^{-2}\cdot{\sqrt[4]{3}}}{3}
  \end{equation}

\end{tabularx}

\newpage
Narysuj wykres funkcji

\begin{tabularx}{\linewidth}{@{}XX@{}XX@{}}

  \begin{equation}
    y=x
  \end{equation}
&
  \begin{equation}
    y=2x
  \end{equation}
&
  \begin{equation}
    y=3x
  \end{equation}
\\
  \begin{equation}
    y=\frac{1}{2}x
  \end{equation}
&
  \begin{equation}
    y=\frac{1}{3}x
  \end{equation}
&
  \begin{equation}
    y=x+1
  \end{equation}
\\
  \begin{equation}
    y=x+2
  \end{equation}
&
  \begin{equation}
    y=x-2
  \end{equation}
&
  \begin{equation}
    y=2x-1
  \end{equation}
\\
  \begin{equation}
    y=3x-2
  \end{equation}
&
  \begin{equation}
    f(x)=2-x
  \end{equation}
&
  \begin{equation}
    f(x)=\frac{1}{2}x-2
  \end{equation}
\\
  \begin{equation}
    y=2x-4
  \end{equation}
&
  \begin{equation}
    y=-x
  \end{equation}
&
  \begin{equation}
    y=-2x
  \end{equation}
\\
  \begin{equation}
    y=-3x-2
  \end{equation}
&
  \begin{equation}
    y=-\frac{1}{2}x-4
  \end{equation}
&
  \begin{equation}
    y=2
  \end{equation}
\\
  \begin{equation}
    g(t)=3t+1
  \end{equation}
&
  \begin{equation}
    x\mapsto x-2
  \end{equation}
&
  \begin{equation}
    t\mapsto -2-2t
  \end{equation}
\\
  \begin{equation}
    f(x)=0
  \end{equation}
&
  \begin{equation}
    2y=4x-2
  \end{equation}
&
  \begin{equation}
    y=x\cdot\sqrt{2}-1
  \end{equation}
\\
\end{tabularx}

\newpage
Rozwiąż równanie

\begin{tabularx}{\linewidth}{@{}XX@{}}

  \begin{equation}
    -3x=4-2\cdot(3x+2)
  \end{equation}
&
  \begin{equation}
    x^2-2x=4+(x+1)^2
  \end{equation}
\\
  \begin{equation}
    x-(1-2x)=3
  \end{equation}
&
  \begin{equation}
    x-(2x+1)^2=-x-4x^2+3
  \end{equation}
\\
  \begin{equation}
    4x+1=3x-\sqrt{2}
  \end{equation}
&
  \begin{equation}
    1-3x=3\sqrt{2}\cdot x-3
  \end{equation}
\\
  \begin{equation}
    2-\frac{1}{2}(x+7)=frac{3-x}{2}
  \end{equation}
&
  \begin{equation}
    \frac{2-x}{3}+\frac{x-4}{2}=1
  \end{equation}
\\
  \begin{equation}
    (x+4)(x+3)=x^2-7
  \end{equation}
&
  \begin{equation}
    \frac{4x-2}{2}-\frac{3x-3}{3}=4-x
  \end{equation}
\\
  \begin{equation}
    \frac{2x+7}{3}=\frac{4x-1}{5}
  \end{equation}
&
  \begin{equation}
    \frac{7-3x}{2}-x+4=3x-1
  \end{equation}
\\
  \begin{equation}
    x(x-3)-x^2=0
  \end{equation}
&
  \begin{equation}
    (x+4)(3-x)=-x^2-(2x+1)
  \end{equation}
\\
\end{tabularx}

\end{document}